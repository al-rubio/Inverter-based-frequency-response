As part of the international efforts set to counteract global warming, the deployment of renewable energies in the electric sector has been considered an energetic priority as a measure to reduce $CO_{2}$ emissions. This objective is also reflected in the regulatory energy policies and plans of some countries. For instance, in Germany, the transformation of the electricity sector contemplates achieving a share of electricity generation of 80\% from renewable energy by 2050. As part of such transformation, the expansion of renewables and the decommissioning of conventional power plants is regulated by the ''Erneuerbare Energien Gesetz” \cite{AgoraEnergiewende.}.\\

Decommissioning conventional power plants and its replacement with inverter-based renewable power plants has as an effect the reduction of system inertia; and consequently the increment of RoCoF values. The relevance of system inertia is to avoid rapid changes in frequency as load-generation imbalances take place; in this way, enough time is given for the activation of the primary power reserve to recover the balanced stable conditions. Currently in Germany, the commitment of renewable energy plants has dispatch priority in the power market due to its zero marginal cost for power generation. This affects market auctions and also technical implications \cite{energiewende2017flexibility}. Balancing of the residual load is provided by conventional units, so curtailment of renewable energy resources is the least preferred option for power balancing \cite{dena2014}. As an immediate result of an imbalance between generation and load, the system frequency starts deviating from its rated value. In continental Europe, the range between 49.8 Hz and 50.2 should be maintained by reserves after a power imbalance. This frequency range corresponds to the ordinary operation range. The primary reserve of the interconnected system can withstand a power imbalance of 3 GW when the system has a total load of 150 GW \cite{ENTSOE.2016}. At the European level, the reference incident case scenario of power loss of 3 GW has been found adequate even with high penetration of renewables \cite{ENTSOE.2016, dena2014}. Nevertheless, there will be still many hours with positive residual load and due to the decommissioning of conventional power plants; their diminished capacity to provide balancing power services at such low inertia levels will have to be compensated by balancing services coming from renewables/storage. Additionally to the uncertainty of conventional generation availability in the German power system, it is also not clear whether instantaneous reserve services from abroad would be available and if transmission capacities will be enough for such \cite{dena2014}.\\

Some ancillary services have been included in the inverter’s capabilities; inverter-based generation from PV has been employed to contribute to voltage regulation by providing reactive power to the grid. Similarly, other approaches have been implemented for over frequency cases, by curtailment implementation and ramping limitation of inverters when the system frequency approaches an upper limit defined by local codes \cite{hoke2018fast, hokefreqwatt}. In the same sense, new techniques have been developed to enable inverter-based generation, such as PV and wind, to also participate in frequency support for under-frequency cases. The most common techniques try to emulate the droop power-frequency characteristic of the synchronous machine by leaving some power headroom during normal operation. Then when a system frequency sag occurs, the inverter can push part of the total available headroom power to counteract the frequency drop \cite{dreidy2017inertia}. Hence, renewable sources are not any longer operating at their maximum power point, these methods also have some economic constraints. Another approach to limit the frequency drop during the seconds after an event leading to a frequency decay is to mimic the inertial response of synchronous machines. Since PV systems do not count with rotating masses, this approach is only achievable with wind turbines and called synthetic or hidden inertia. Due to the decoupling of wind turbines from the grid dynamics, modified control strategies in the power electronics allows the controller to extract part of the stored kinetic energy in the rotating masses of the wind turbine by adjusting the electromagnetic torque in the generator \cite{dreidy2017inertia}. \\

Although the integration of more inverter-based generation causes higher values of RoCoF, they also present the solution with the implementation of fast power reserve for frequency support. Whereas synchronous power reserve deployment is in the order of few seconds (5-30s), power electronics implementation offer full power deployment in the order of milliseconds \cite{miller2017technology}. In this investigation, the conditions which should be fulfilled by inverters in highly penetrated grids by non-synchronous generation, to provide an inverter-based fast power reserve (IBFPR) are investigated. Then the required triggering time and power response to avoid under-frequency load shedding (UFLS) are estimated. The over-frequency phenomenon is treated with the same approach as the under-frequency case. The effectiveness of synthetic inertia is evaluated under some assumed future scenario conditions. Two grid cases are utilized to assess the influence of the grid size, synchronous response and common simplifications made in power system analysis. The IEEE 9 bus benchmark grid model and an electric power system in the European scale are considered for such purpose; a methodology to determine the requirements of the fleet of inverters to offer frequency support is developed.
