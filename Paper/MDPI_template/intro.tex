
  As part of the international efforts set to counteract global warming, the deployment of renewable energies in the electric sector has been considered an energetic priority as a measure to reduce $CO_{2}$ emissions. This objective is also reflected in the regulatory energy policies and plans of some countries. For instance in Germany the transformation of the electricity sector through renewable energies, known as ''Energiewende”, contemplates to achieve a share in electricity consumption from renewables of 80\% by 2050 [3]. As part of it, the renewable energy act, ''Erneuerbare Energien Gesetz”, regulates the expansion of renewables and decommissioning of convectional generation. \\
    
 Decommissioning of convectional power plants and its replacement with inverter-based renewable power plants has as an effect, the reduction of system inertia and consequently the increment in values of RoCoF. The relevance of system inertia is to avoid rapid changes in frequency as load-generation imbalance takes place; in this way, enough time is given to the activation of the primary power reserve to recover the balanced stable conditions[ref].\\ %Therefore, the need of new frequency control strategies is evident in this context. Due to the expected higher values of RoCoF, load shedding caused by low frequency and generation curtailment due to over-frequency may occur faster than nowadays grid configuration[ref]. Hence, the participation of non-synchronous generation in providing frequency support as ancillary service is a key factor in achieving high integration of renewables without jeopardizing power system reliability. \\ 
 Currently in Germany, plant commitment of renewable energy plants have a priority in the power market for dispatch due to its zero marginal cost for generation. This has effect in market auctions and also technical implications [12]. Balancing of the residual load is provided by conventional units, so curtailment of renewable energy resources is the last preferred option for power balancing [16].\\
   
  As an immediate result of an imbalance between generation and load, the system frequency starts deviating from its rated value. The range of 49.8 Hz and 50.2, in continental Europe should be maintained by reserves after a power imbalance. This frequency range corresponds to the ordinary operation range. The primary reserve of the interconnected system is able to withstand a power imbalance of 3 GW when the system has a total load of 150 GW [1].
  At an European level, the reference incident case scenario of power loss of 3 GW has been found adequate even with high penetration of renewables [1, 16]. Nevertheless, there still will be  many hours with positive residual load and due to the decommissioning of conventional power plants [16, 17]; their diminished capacity to provide balancing power services at such low inertia levels will have to be compensated by balancing services coming from renewables/storage. Additionally to the uncertainty of conventional generation availability in the German power system, is also not clear whether instantaneous reserve services from abroad would be available and if transmission capacities will be enough for such [16].\\  
 
 Some ancillary services have been included in the inverter’s capabilities; inverter based generation from PV has been employed to contribute in voltage regulation by means of providing reactive power to the grid. Similarly, other approaches have been implemented for over frequency cases, by curtailment implementation and ramping limitation of inverters when the system's frequency approaches an upper limit defined by local codes [4, 5]. In the same sense, new techniques have been developed in order to enable inverter based generation, such as PV and wind, to also participate in frequency support for under-frequency cases. The most common techniques try to emulate the droop power-frequency characteristic of the synchronous machine by leaving some power headroom during normal operation, so when a system frequency sag occurs, the inverter is able to push part or the total available headroom power in order to counteract the frequency drop [6]. Hence renewable sources are not any longer operating at its maximum power point, these methods also have some economic constraints. Another approach to limit the frequency drop during the seconds after an event leading to a frequency decay, is to mimic the inertial response of synchronous machines. Since PV systems do not count with rotating masses, this approach is only achievable with wind turbines and called synthetic or hidden inertia [ref]. Due to the decoupling of wind turbines from the grid dynamics, modified control strategies in the power electronics allows the controller to extract part of the stored kinetic energy in the rotating masses of the wind turbine by adjusting the electromagnetic torque in the generator [6]. \\
 
 Although the integration of more inverter based generations causes higher values of RoCoF, they also present the solution with the implementation of fast power reserve for frequency support. Whereas synchronous power reserve deployment is in the order of few seconds (5-30s), power electronics implementation offer full power deployment in the order of miliseconds [ref]. Table \ref{tb:tech_times} lists some important and typical time scales of the most common power electronic technologies implemented in modern power systems. In this investigation the conditions, which should be fulfilled by inverters in highly penetrated grids by non-synchronous generation, to provide an inverter based fast power reserve (IBFPR) are investigated. Then the required triggering time and power response to avoid under-frequency load shedding (UFLS) are estimated. Over-frequency phenomenon is treated with the same approach as the under-frequency case. The effectiveness of synthetic inertia is evaluated under some assumed future scenario conditions. Two grid cases are utilized in order to assess the influence of the grid size, synchronous response and common simplifications made in power system analysis. The IEEE 9 bus benchmark grid model and an electric power system in the European scale are considered for such purpose; a methodology to determine the requirements of the fleet of inverters to offer frequency support is developed.\\
 
\begin{table}[h]
	\caption{\label{tb:tech_times}: Activation times for fast power reserve of non-synchronous technologies [14]}
	\centering
	%% \tablesize{} %% You can specify the fontsize here, e.g., \tablesize{\footnotesize}. If commented out \small will be used.
	\begin{tabular}{cc}
		\toprule
		\textbf{Technology} 	& \textbf{ Full Fast Frequency response (ms)}\\
		\midrule
			
		Wind turbine-Synthetic inertia&	$ \sim500 $\\
		Lithium batteries&	10-20\\
		Flow batteries&	10-20\\
		Lead-acid batteries&	40\\
		Flywheels&	<4\\
		Super capacitor&	10-20\\
		Solar PV&	100-200\\
		HVDC&	50-500\\

		\bottomrule
	\end{tabular}
\end{table}
 
 

 
 
 %The subsequent chapters will be covering the following content: Chapter 2 corresponds to the literature related to power system operation and primary reserve. Here an overview to the aspects behind the operation of typical power systems is presented, as well as how frequency is stablished and controlled. Additionally, the influence and contribution from inverter based generators is noted. Chapter 3 presents the methodology to estimate the power rate needed to avoid transient frequency instability. The developed method and expression for power rate are then evaluated in two cases: A micro-grid with fast primary reserve response and an electric island in a European scale with relative slow primary reserve response. A detailed demonstration of parameter setting and assumptions is presented as well. The simulation results of the implemented method in each scenario is then shown in Chapter 4; critical times, frequency nadir and power responses are obtained. A detailed discussion and analysis of the main features and trends observed in the results section, is then followed in Chapter 5. The main conclusions and areas for future work are outlined in Chapters 6 and 7 respectively.

