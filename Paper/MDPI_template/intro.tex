 \section{}
  As part of the international efforts set to counteract global warming, the deployment of renewable energies in the electric sector has been considered an energetic priority as a measure to reduce $CO_{2}$ emissions. This objective is also reflected in the regulatory energy policies and plans of some countries. For instance, in Germany the transformation of the electricity sector through renewables, known as ''Energiewende”, contemplates to achieve a share in electricity consumption from renewables of 80 percent [3]. As part of it, the renewable energy act, ''Erneuerbare Energien Gesetz”, regulates the expansion of renewables and convectional generation decommissioning. \\
 
 Even though power systems have grown in size and complexity, frequency control has been always performed through power balancing between generation and demand due to synchronous generator characteristics. The variation of load during a given period of time is followed by a change on the prime mover power of the synchronous generator. When an imbalance occurs, the excess or lack of power is injected to or released from the kinetic energy in the rotor. Therefore, the magnitude of the rate of change of frequency (RoCoF) during an imbalance is inversely proportional to the system’s inertia. 
 
 Decommissioning of convectional generating power plants and its replacement with inverter-based renewables power plants has as an effect a reduction of system inertia and consequently increasing values of RoCoF. The relevance of system inertia is to avoid rapid changes in frequency as load-generation imbalance takes place; in this way, enough time is given to the activation of the primary power reserve to recover the balanced stable conditions. Therefore, the need of new frequency control strategies is evident in this context. Due to the expected higher values of RoCoF, load shedding caused by low frequency and generation curtailment due to over-frequency may occur faster than nowadays grid configuration. Hence, the participation of non-synchronous generation in providing frequency support as ancillary service is a key factor in achieving high integration of renewables without jeopardizing power system reliability. 
 
 So far, some ancillary services have been included in inverter’s capabilities; inverter based generation from PV has been employed to contribute in voltage regulation by meanings of providing reactive power to the grid. Similarly, other approaches have been implemented for over frequency cases, by curtailment implementation and ramping limitation of inverters when system frequency approaches an upper limit allowed by local codes [4, 5]. In the same sense, new techniques have been developed in order to enable inverter based generation, such as PV and Wind, to also participate in frequency support for under-frequency cases. The most common techniques try to emulate the droop power-frequency characteristic of the synchronous machine by leaving some power headroom during normal operation, so when a system frequency sag occurs, the inverter is able to push part or the total available power headroom to counteract the frequency drop [6]. Hence renewable sources are not any longer operating at its maximum power point, therefore these methods also have some economic constraints. Another approach to limit the frequency drop during the seconds after an event leading to a frequency decay, is to mimic the inertial response of synchronous machines. Since PV systems do not count with rotating masses, this approach is only achievable with wind turbines and called synthetic or hidden inertia ref. Due to the decoupling of wind turbines from the grid dynamics, modified control strategies in the power electronics allows the controller to extract part of the stored kinetic energy in the rotating masses of the wind turbine [6]. 
 In this investigation the conditions, which should be fulfilled by inverters in highly penetrated grids by non-synchronous generation, to provide an inverter based fast power reserve (IBFPR) are investigated. Then the required triggering time and power response to avoid under-frequency load shedding (UFLS) in such kind of islanded grids are estimated. Over-frequency phenomenon is treated with the same approach as the under-frequency case. The alternative of synthetic inertia is evaluated to evaluated its effectiveness under some assumed future scenario conditions. With the consideration of two cases, a benchmark grid model is taken as a small scale grid and an electric island in the European scale; a methodology to determine the requirements of the fleet of inverters to offer frequency support is developed. %The subsequent chapters will be covering the following content: Chapter 2 corresponds to the literature related to power system operation and primary reserve. Here an overview to the aspects behind the operation of typical power systems is presented, as well as how frequency is stablished and controlled. Additionally, the influence and contribution from inverter based generators is noted. Chapter 3 presents the methodology to estimate the power rate needed to avoid transient frequency instability. The developed method and expression for power rate are then evaluated in two cases: A micro-grid with fast primary reserve response and an electric island in a European scale with relative slow primary reserve response. A detailed demonstration of parameter setting and assumptions is presented as well. The simulation results of the implemented method in each scenario is then shown in Chapter 4; critical times, frequency nadir and power responses are obtained. A detailed discussion and analysis of the main features and trends observed in the results section, is then followed in Chapter 5. The main conclusions and areas for future work are outlined in Chapters 6 and 7 respectively.

