%  LaTeX support: latex@mdpi.com
%  In case you need support, please attach all files that are necessary for compiling as well as the log file, and specify the details of your LaTeX setup (which operating system and LaTeX version / tools you are using).

%=================================================================
\documentclass[energies,article,submit,moreauthors,pdftex]{Definitions/mdpi}

%\bibliography{Thesisref_LATEX.bib}
% If you would like to post an early version of this manuscript as a preprint, you may use preprint as the journal and change 'submit' to 'accept'. The document class line would be, e.g., \documentclass[preprints,article,accept,moreauthors,pdftex]{mdpi}. This is especially recommended for submission to arXiv, where line numbers should be removed before posting. For preprints.org, the editorial staff will make this change immediately prior to posting.

%--------------------
% Class Options:
%--------------------
%----------
% journal
%----------
% Choose between the following MDPI journals:
% acoustics, actuators, addictions, admsci, aerospace, agriculture, agriengineering, agronomy, algorithms, animals, antibiotics, antibodies, antioxidants, applsci, arts, asc, asi, atmosphere, atoms, axioms, batteries, bdcc, behavsci , beverages, bioengineering, biology, biomedicines, biomimetics, biomolecules, biosensors, brainsci , buildings, cancers, carbon , catalysts, cells, ceramics, challenges, chemengineering, chemistry, chemosensors, children, cleantechnol, climate, clockssleep, cmd, coatings, colloids, computation, computers, condensedmatter, cosmetics, cryptography, crystals, dairy, data, dentistry, designs , diagnostics, diseases, diversity, drones, econometrics, economies, education, ejihpe, electrochem, electronics, energies, entropy, environments, epigenomes, est, fermentation, fibers, fire, fishes, fluids, foods, forecasting, forests, fractalfract, futureinternet, futurephys, galaxies, games, gastrointestdisord, gels, genealogy, genes, geohazards, geosciences, geriatrics, hazardousmatters, healthcare, heritage, highthroughput, horticulturae, humanities, hydrology, ijerph, ijfs, ijgi, ijms, ijns, ijtpp, informatics, information, infrastructures, inorganics, insects, instruments, inventions, iot, j, jcdd, jcm, jcp, jcs, jdb, jfb, jfmk, jimaging, jintelligence, jlpea, jmmp, jmse, jnt, jof, joitmc, jpm, jrfm, jsan, land, languages, laws, life, literature, logistics, lubricants, machines, magnetochemistry, make, marinedrugs, materials, mathematics, mca, medicina, medicines, medsci, membranes, metabolites, metals, microarrays, micromachines, microorganisms, minerals, modelling, molbank, molecules, mps, mti, nanomaterials, ncrna, neuroglia, nitrogen, notspecified, nutrients, ohbm, optics, particles, pathogens, pharmaceuticals, pharmaceutics, pharmacy, philosophies, photonics, physics, plants, plasma, polymers, polysaccharides, preprints , proceedings, processes, proteomes, psych, publications, quantumrep, quaternary, qubs, reactions, recycling, religions, remotesensing, reports, resources, risks, robotics, safety, sci, scipharm, sensors, separations, sexes, signals, sinusitis, smartcities, sna, societies, socsci, soilsystems, sports, standards, stats, surfaces, surgeries, sustainability, symmetry, systems, technologies, test, toxics, toxins, tropicalmed, universe, urbansci, vaccines, vehicles, vetsci, vibration, viruses, vision, water, wem, wevj

%---------
% article
%---------
% The default type of manuscript is "article", but can be replaced by:
% abstract, addendum, article, benchmark, book, bookreview, briefreport, casereport, changes, comment, commentary, communication, conceptpaper, conferenceproceedings, correction, conferencereport, expressionofconcern, extendedabstract, meetingreport, creative, datadescriptor, discussion, editorial, essay, erratum, hypothesis, interestingimages, letter, meetingreport, newbookreceived, obituary, opinion, projectreport, reply, retraction, review, perspective, protocol, shortnote, supfile, technicalnote, viewpoint
% supfile = supplementary materials

%----------
% submit
%----------
% The class option "submit" will be changed to "accept" by the Editorial Office when the paper is accepted. This will only make changes to the frontpage (e.g., the logo of the journal will get visible), the headings, and the copyright information. Also, line numbering will be removed. Journal info and pagination for accepted papers will also be assigned by the Editorial Office.

%------------------
% moreauthors
%------------------
% If there is only one author the class option oneauthor should be used. Otherwise use the class option moreauthors.

%---------
% pdftex
%---------
% The option pdftex is for use with pdfLaTeX. If eps figures are used, remove the option pdftex and use LaTeX and dvi2pdf.

%=================================================================
\firstpage{1}
\makeatletter
\setcounter{page}{\@firstpage}
\makeatother
\pubvolume{xx}
\issuenum{1}
\articlenumber{5}
\pubyear{2019}
\copyrightyear{2019}
%\externaleditor{Academic Editor: name}
\history{Received: date; Accepted: date; Published: date}
%\updates{yes} % If there is an update available, un-comment this line

%% MDPI internal command: uncomment if new journal that already uses continuous page numbers
%\continuouspages{yes}

%------------------------------------------------------------------
% The following line should be uncommented if the LaTeX file is uploaded to arXiv.org
%\pdfoutput=1

%=================================================================
% Add packages and commands here. The following packages are loaded in our class file: fontenc, calc, indentfirst, fancyhdr, graphicx, lastpage, ifthen, lineno, float, amsmath, setspace, enumitem, mathpazo, booktabs, titlesec, etoolbox, amsthm, hyphenat, natbib, hyperref, footmisc, geometry, caption, url, mdframed, tabto, soul, multirow, microtype, tikz
\usepackage{graphicx}
\usepackage{caption}
\usepackage{subcaption}
\graphicspath{{img/}}
\usepackage{booktabs}
\usepackage{amsmath}
\usepackage{footnote}
\usepackage{steinmetz}
\usepackage{adjustbox}
%\makesavenoteenv{tabular}
%\makesavenoteenv{table}
\usepackage{fixltx2e}
%\externalbibliography{yes}
%\usepackage{biblatex}
%\addbibresource{Thesisref_LATEX.bib}
%\bibliography{Thesisref_LATEX.bib}
%=================================================================
%% Please use the following mathematics environments: Theorem, Lemma, Corollary, Proposition, Characterization, Property, Problem, Example, ExamplesandDefinitions, Hypothesis, Remark, Definition, Notation, Assumption
%% For proofs, please use the proof environment (the amsthm package is loaded by the MDPI class).

%=================================================================
% Full title of the paper (Capitalized)
\Title{Determination of the required Power Response of Inverters to provide fast Frequency Support in Power Systems with low Synchronous Inertia}

% Author Orchid ID: enter ID or remove command
%\newcommand{\orcidauthorA}{0000-0000-000-000X} % Add \orcidA{} behind the author's name
%\newcommand{\orcidauthorB}{0000-0000-000-000X} % Add \orcidB{} behind the author's name

% Authors, for the paper (add full first names)
\Author{Alejandro Rubio $^{1,*}, Holger Behrends $^{1}$, Stefan Geißendörfer $^{1}$, Karsten von Maydell, Carsten Agert}

% Authors, for metadata in PDF
\AuthorNames{Alejandro Rubio, Holger Behrends and Stefan Geißendörfer}

% Affiliations / Addresses (Add [1] after \address if there is only one affiliation.)
\address[1]{%
$^{1}$ \quad DLR Institute of Networked Energy Systems, Carl-von-Ossietzky-Str. 15, 26129 Oldenburg, Germany; holger.behrends@dlr.de (H.B.); stefan.geissendoerfer@dlr.de (S.G.); karsten.maydell@dlr.de (K.M.); carsten.agert@dlr.de (C.A.) \\
%$^{2}$ \quad Affiliation 2; e-mail@e-mail.com}

% Contact information of the corresponding author
\corres{Correspondence: alejandro.rubio@dlr.de; Tel.: +49-441-99906-481}

% Current address and/or shared authorship
\firstnote{Current address: Affiliation 3}
\secondnote{These authors contributed equally to this work.}
% The commands \thirdnote{} till \eighthnote{} are available for further notes

%\simplesumm{} % Simple summary

%\conference{} % An extended version of a conference paper

% Abstract (Do not insert blank lines, i.e. \\)
\abstract{Decommissioning of conventional power plants and the installation of inverter-based renewable energy technologies decrease the overall power system's inertia, increasing the rate of change of frequency of the system (RoCoF). These expected high values of RoCoF shorten the time response needed before load shedding or generation curtailment takes place. In a future scenario where renewables are predominant in power systems, the ability of synchronous machines to meet such conditions is uncertain in terms of capacity and time response. The implementation of fast power reserve and synthetic inertia from inverter-based sources was assessed through the simulation of two scenarios with different grid sizes and primary reserve response. As main results it was obtained that full activation time for fast power reserve with penetration above 80\% of inverter-based generation would need to be 100 ms or less for imbalances up to 40\% regardless of the synchronous response and grid size; meaning that current frequency measurement techniques and the time for fast power reserve deployment would not ensure system stability under high unbalanced conditions. At less unbalanced conditions, the grid in the European scale was found to become critical with imbalances starting at 3\% and non-synchronous share of 60\%.}

% Keywords
\keyword{fast power reserve, frequency nadir, critical time, low inertia grids}

% The fields PACS, MSC, and JEL may be left empty or commented out if not applicable
%\PACS{J0101}
%\MSC{}
%\JEL{}

%%%%%%%%%%%%%%%%%%%%%%%%%%%%%%%%%%%%%%%%%%
% Only for the journal Diversity
%\LSID{\url{http://}}

%%%%%%%%%%%%%%%%%%%%%%%%%%%%%%%%%%%%%%%%%%
% Only for the journal Applied Sciences:
%\featuredapplication{Authors are encouraged to provide a concise description of the specific application or a potential application of the work. This section is not mandatory.}
%%%%%%%%%%%%%%%%%%%%%%%%%%%%%%%%%%%%%%%%%%

%%%%%%%%%%%%%%%%%%%%%%%%%%%%%%%%%%%%%%%%%%
% Only for the journal Data:
%\dataset{DOI number or link to the deposited data set in cases where the data set is published or set to be published separately. If the data set is submitted and will be published as a supplement to this paper in the journal Data, this field will be filled by the editors of the journal. In this case, please make sure to submit the data set as a supplement when entering your manuscript into our manuscript editorial system.}

%\datasetlicense{license under which the data set is made available (CC0, CC-BY, CC-BY-SA, CC-BY-NC, etc.)}

%%%%%%%%%%%%%%%%%%%%%%%%%%%%%%%%%%%%%%%%%%
% Only for the journal Toxins
%\keycontribution{The breakthroughs or highlights of the manuscript. Authors can write one or two sentences to describe the most important part of the paper.}

%\setcounter{secnumdepth}{4}
%%%%%%%%%%%%%%%%%%%%%%%%%%%%%%%%%%%%%%%%%%
\begin{document}
%%%%%%%%%%%%%%%%%%%%%%%%%%%%%%%%%%%%%%%%%%

\section{Introduction}
\label{sec:intro}
As part of the international efforts set to counteract global warming, the deployment of renewable energies in the electric sector has been considered an energetic priority as a measure to reduce $CO_{2}$ emissions. This objective is also reflected in the regulatory energy policies and plans of some countries. For instance in Germany the transformation of the electricity sector through renewable energies, known as ''Energiewende”, contemplates to achieve a share in electricity consumption from renewables of 80\% by 2050 \cite{AgoraEnergiewende.}. As part of it, the renewable energy act, ''Erneuerbare Energien Gesetz”, regulates the expansion of renewables and decommissioning of convectional generation.\\

Decommissioning of conventional power plants and its replacement with inverter-based renewable power plants has as an effect, the reduction of system inertia and consequently the increment in values of RoCoF. The relevance of system inertia is to avoid rapid changes in frequency as load-generation imbalance takes place; in this way, enough time is given to the activation of the primary power reserve to recover the balanced stable conditions. Currently in Germany, plant commitment of renewable energy plants have a priority in the power market for dispatch due to its zero marginal cost for generation. This has effect in market auctions and also technical implications \cite{energiewende2017flexibility}. Balancing of the residual load is provided by conventional units, so curtailment of renewable energy resources is the last preferred option for power balancing \cite{dena2014}. As an immediate result of an imbalance between generation and load, the system frequency starts deviating from its rated value. The range of 49.8 Hz and 50.2, in continental Europe should be maintained by reserves after a power imbalance. This frequency range corresponds to the ordinary operation range. The primary reserve of the interconnected system is able to withstand a power imbalance of 3 GW when the system has a total load of 150 GW \cite{ENTSOE.2016}. At an European level, the reference incident case scenario of power loss of 3 GW has been found adequate even with high penetration of renewables \cite{ENTSOE.2016, dena2014}]. Nevertheless, there still will be  many hours with positive residual load and due to the decommissioning of conventional power plants [16, 17]; their diminished capacity to provide balancing power services at such low inertia levels will have to be compensated by balancing services coming from renewables/storage. Additionally to the uncertainty of conventional generation availability in the German power system, is also not clear whether instantaneous reserve services from abroad would be available and if transmission capacities will be enough for such \cite{dena2014}.\\

Some ancillary services have been included in the inverter’s capabilities; inverter based generation from PV has been employed to contribute in voltage regulation by means of providing reactive power to the grid. Similarly, other approaches have been implemented for over frequency cases, by curtailment implementation and ramping limitation of inverters when the system's frequency approaches an upper limit defined by local codes \cite{hoke2018fast, hokefreqwatt}. In the same sense, new techniques have been developed in order to enable inverter based generation, such as PV and wind, to also participate in frequency support for under-frequency cases. The most common techniques try to emulate the droop power-frequency characteristic of the synchronous machine by leaving some power headroom during normal operation, so when a system frequency sag occurs, the inverter is able to push part or the total available headroom power in order to counteract the frequency drop \cite{dreidy2017inertia}. Hence renewable sources are not any longer operating at its maximum power point, these methods also have some economic constraints. Another approach to limit the frequency drop during the seconds after an event leading to a frequency decay, is to mimic the inertial response of synchronous machines. Since PV systems do not count with rotating masses, this approach is only achievable with wind turbines and called synthetic or hidden inertia. Due to the decoupling of wind turbines from the grid dynamics, modified control strategies in the power electronics allows the controller to extract part of the stored kinetic energy in the rotating masses of the wind turbine by adjusting the electromagnetic torque in the generator \cite{dreidy2017inertia}. \\

Although the integration of more inverter based generations causes higher values of RoCoF, they also present the solution with the implementation of fast power reserve for frequency support. Whereas synchronous power reserve deployment is in the order of few seconds (5-30s), power electronics implementation offer full power deployment in the order of miliseconds \cite{miller2017technology}. In this investigation the conditions, which should be fulfilled by inverters in highly penetrated grids by non-synchronous generation, to provide an inverter based fast power reserve (IBFPR) are investigated. Then the required triggering time and power response to avoid under-frequency load shedding (UFLS) are estimated. Over-frequency phenomenon is treated with the same approach as the under-frequency case. The effectiveness of synthetic inertia is evaluated under some assumed future scenario conditions. Two grid cases are utilized in order to assess the influence of the grid size, synchronous response and common simplifications made in power system analysis. The IEEE 9 bus benchmark grid model and an electric power system in the European scale are considered for such purpose; a methodology to determine the requirements of the fleet of inverters to offer frequency support is developed.
%%%%%%%%%%%%%%%%%%%%%%%%%%%%%%%%%%%%%%%%%%
\section{Methodology}
\label{sec:method}
\subsection{Frequency limits and Inertia Constants}
When the global security of the system is endangered and under/above frequency is experienced then load shedding is activated; the system is said to be in the emergency state. If the frequency exceeds the range of 47.5 Hz or 51.5 Hz, a system blackout can hardly be avoided \cite{ENTSOE.2016}. Consequently the system will reach the so-called blackout state and will have to be restored. Before black out the system tries to recover balance by rejecting partial load starting at 49 Hz as frequency decreases. On the other hand, curtailment thresholds between 50.2 and 50.5 have been studied by ENTSOE for over-frequency scenarios \cite{ENTSOE.2016}. In this research, a deviation of $ \pm1 $ Hz is used as threshold before load shedding and curtailment starts. Hence, to keep frequency within such threshold; the investigated critical time and power response, correspond to the maximum allowed time for fast power reserve activation in order to inject power from renewables or storage in case of under-frequency; or to extract power from the grid to be stored or converter in another energy form in the case of over-frequency.

Two terms commonly found in the literature of power system stability will be used along this section:

\begin{itemize}[leftmargin=*,labelsep=5.8mm]
	\item \textbf{Inertia constant (H)}: It has units of seconds (s) and it is the ratio of the stored kinetic energy in the rotating masses of the machine ($E_k$ in MWs) and its nominal capacity ($S_{nom}$ in MVA).\\
	\item \textbf{Acceleration time constant (T\textsubscript{a})}: It also has the units of seconds (s) but this is the ratio of double the kinetic energy (MWs) and generators nominal power output ($P_{nom}$ in MW).\\
	 Acceleration time constant is a measure of the robustness before disturbances of the system. It could be interpreted as the required time to remove the kinetic energy from the rotating masses of the generators connected in a grid at the rate of the supplied power load. Hence, the higher the time constant, the higher the kinetic energy available. As the share of synchronous generations decreases, this constant decreases proportionally.
\end{itemize}

With $f$ as frequency, $f_0$ as nominal frequency and $\Delta P$ as power imbalance, the swing equation can be expressed as follows \cite{kundur1994power}:
%Equation 3 1
\begin{equation}
	\label{eq:swing}
	\frac{df}{dt}=\dfrac{\Delta P*f_0}{2*H*S_{nom}}=\frac{\Delta P*f_0}{T_a*P_{nom}}=\frac{\Delta P*f_0}{2*E_k}
\end{equation}

In this paper, the inertia constant $ H $ is used for the description of inertia in wind turbines and single synchronous machine representation whereas the system acceleration constant $ T_a $ is used to express the whole system inertia related to the load in terms of real power.

\subsection{Frequency Support from Inverter based Generation}

In this section the methodology and considerations for the implementation of inverter based generation for frequency support are explained.
 
\subsubsection{Synthetic Inertia}

Synthetic inertia is one of the techniques that manufactures and researchers are considering to tackle with the low inertia problem in power systems \cite{Gevorgian.2017, GeneralElectricInternational.2013}. Frequency support through synthetic inertia was considered with the following assumptions \cite{dreidy2017inertia, nesje2015need}:
\begin{enumerate}[leftmargin=*,labelsep=4.9mm]
	\item Power output from synthetic inertia is limited to 10\% of the wind turbine nominal power.
	\item Due to mechanical and thermal stresses, the additional power can be delivered only for a maximum time of 10 s.
	\item It is assumed that all wind turbines operate at its nominal power output. The value of 1.5 MW was selected for such purpose.
	\item The maximum allowable amount of kinetic energy to be extracted from the turbines was limit to half of the kinetic energy while the turbine operates at nominal speed \cite{NREL.2012}.

\end{enumerate}

	A control system is needed so the stored energy in the rotating blades can be extracted from the wind turbine. Using the expression of power as the derivative of the stored energy in the blades Equation \eqref{eq:si} is obtained. The additional extracted power from the wind turbine through the implementation of Equation \eqref{eq:si} accounts for the synthetic inertia contribution \cite{NREL.2012}.

\begin{equation}
	\label{eq:si}
	P_{pu}(t)=2*H_{wt}*\omega_{pu}(t)*\dfrac{d\omega_{pu}(t)}{dt}
\end{equation}

%Equation 3 4

Where  $H_{wt}$ is the turbine inertia constant and $\omega_{pu}$ the rotational speed in per unit.

\begin{figure}[h]
	\centering
	\includegraphics[width=0.8\textwidth]{/method/SI2}
	\caption{Representation of Equation \eqref{eq:si} in Simulink. In the figure it can be seen the insertion of a filter at the output of the multiplication block \cite{GeneralElectricInternational.2013, nesje2015need}. A constant block $ K_i $ adjusts the initial response in the model. Since Equation \eqref{eq:si} is given in pu, the output is multiplied by a constant $ P_{wt} $ representing the rated power of the turbine.}
	\label{fig:synthetic}
\end{figure}

Typical values of inertia constant for wind turbines are not openly available from the manufacturers to the public. An approximate value was calculated with the utilization of an equation which relates nominal power and inertia constant for wind turbines \cite{GonzalezRodriguez.2007}.

\begin{equation}
	\label{eq:wtinertia}
	H_{wt}\approx1.87*P_{nwt}^{0.0597}
\end{equation}


For a wind turbine with nominal power output of 1.5 MW the value of $ H $ corresponds to 4.37 s \cite{Wu.2013}. A rated rotational speed of 18 rev/min was considered \cite{Wu.2013}. To avoid the wind turbine to stall, a reduction of 5 rev/min is allowed by the implementation of the control system. This change of rotational speed equals a reduction of 3 MWs on kinetic energy out of a total of 6 MWs.

%Table 3 1: Constants for implementation of synthetic inertia.
\begin{table}[h]
	\caption{\label{tb:inertia}: Constants for the implementation of synthetic inertia in Simulink, $ n_{wt} $ represents the number of wind turbines with synthetic inertia control}
	\centering
	%% \tablesize{} %% You can specify the fontsize here, e.g., \tablesize{\footnotesize}. If commented out \small will be used.
	\begin{tabular}{cccc}
		\toprule
		\textbf{T\textsubscript{wt}} 	& \textbf{ H\textsubscript{wt} (s)}	& \textbf{ P\textsubscript{wt} (MW)}  & \textbf{ K\textsubscript{i}} \\
		\midrule
			1	       & 4.37		        &  1.5*$ n_{wt} $ & 10 \\
		%entry 2		& data			& data\\
		\bottomrule
	\end{tabular}
\end{table}

\subsubsection{Inverter based fast Power Reserve}

When a power system is subjected to a negative power imbalance and it is assumed that no load is rejected at UFLS frequency, this continues dropping below 49 Hz. The time at which the system frequency equals the UFLS value is then called critical time. This is the maximum available time for the inverter based reserve to deploy the required power to the system.

\begin{figure}[h]
	\centering
	\begin{subfigure}[h]{0.45\textwidth}
		\centering
		\includegraphics[width=\textwidth]{method/fig1}
		\caption{Without IBFPR}
		\label{fig:freqresp_before}
	\end{subfigure}
	\hfill
	\begin{subfigure}[h]{0.45\textwidth}
		\centering
		\includegraphics[width=\textwidth]{method/fig2}
		\caption{With IBFPR}
		\label{fig:freqresp_after}
	\end{subfigure}


	\caption{In (\textbf{a}) the frequency response goes below the 49 Hz leading to UFLS at the critical time, whereas in (\textbf{b}) the IBFPR is applied avoiding ULFS. In this case the power imbalance is compensated at the critical time by the inverters.}
\end{figure}

In the critical condition that would lead to load shedding, it is expected from the IBFPR to at least counteract the RoCoF at the critical time, as illustrated in Figure \ref{fig:freqresp_after}.
Recalling Equation \eqref{eq:swing}; it is necessary that the machine’s accelerating power (power imbalance) become zero at the critical time.
\begin{equation}
	\label{eq:powerbalance}
	P_a (t_{cr} )=P_{mech}-P_{elec}+P_{IBFPR}=0
\end{equation}

Where $ P_a $ is accelerating power, $ P_{mech} $ is mechanical power, $ P_{elec} $ is electrical power load, $ t_{cr} $ is the critical time and $ P_{IBFPR} $ is inverter based fast power reserve.

From the assumption of a linear mechanical power deployment of the synchronous machines governors, the rate of change in mechanical power, after a power imbalance $ \Delta P $, is given by $ \Delta P/t_{nadir} $, where $ t_{nadir} $ represents the time at which the frequency nadir occurs. Given the power balance at the critical time, $ t_{cr} $; the IBFPR response must be equal to $ P_{elec}-P_{mech} $, being $ P_{elec} $ equal to $ \Delta P $.


Substituting $ P_{mech} $ by $ \Delta P* t_{cr} /t_{nadir} $ and $ P_{elec} $ by $ \Delta P $ in Equation \eqref{eq:powerbalance}, the following expression is obtained for the $ P_{IBFPR} $ at time $ t_{cr} $:
%Equation 3 2
%[
\begin{equation}
	\label{eq:p_at_tcr}
	P_{IBFPR} (t_{cr} )=\Delta P*(1-t_{cr}/t_{nadir} )
\end{equation}
It is assumed that $ P_{IBFPR} $ remains with a constant power output after $ t_{cr} $ long enough to stabilize the system frequency. The result of the previous equation represents the slope of the power output since the inception of the incident until the critical time, which with the implementation of IBFPR, it will be not any longer critical but rather it will be the new desired frequency nadir time.
%Equation 3 3: IBFPR before critical time.
\begin{equation}
	\label{eq:IBFPR}
	P_{IBFPR} (t)=\dfrac{\Delta P*(1-t_{cr}/t_{nadir} )*t}{t_{cr}}
\end{equation}

According to the obtained expression in Equation \eqref{eq:IBFPR}; it can be realized that the desired power response from the inverters depends exclusively on parameters which cannot be directly measured from the grid connecting point. In a real situation the values of $\Delta P$, $ t_{nadir} $ and $ t_{cr} $ cannot be known in advance, representing this factors a challenge in the implementation of this ideal power response. Those values are dependent on the grid characteristics, the primary conventional reserve deployment time and the overall system inertia \cite{orum2015future}. Thus two main cases are considered for the remaining analysis with the intent of covering a wider range of systems with different characteristics and dimensions.

\subsection{Simulation Cases}

As presented in the previous section, the values of critical time and frequency nadir depend on the system imbalance and primary reserve deployment time. In spite of assessing the influence of the grid size and the primary reserve characteristics, two main cases are considered. In both cases is assumed that the initial steady frequency is the nominal 50 Hz.


\begin{itemize}[leftmargin=*,labelsep=5.8mm]
	\item \textbf{Small scale grid case:} For the evaluation a well-known and studied benchmark grid topology as the WSCC model, also known as the IEEE 9 bus model is considered. Synchronous reserve deployment is in the order of a few seconds due to governor response \cite{kundur1994power, sundaram2008comparing}. In order to assess the typical simplifications made in power system analysis, two approaches of this cases were developed:
	\subitem Scenario A - Simplified Model: The power system is represented by an equivalent single machine model in which losses are neglected. In this case, typical governor data is considered. It is investigated the critical time for inverters' activation and the required IBFPR is  also determined. Furthermore, the impact of synthetic inertia is analyzed.
	\subitem Scenario B - Extended Model: All the power system components (transmission lines, transformers, exciters and governors of the three generators) and its dynamic characteristics are considered in the IEEE 9 bus model for critical time and IBFPR estimation.\\
	\item \textbf{Large scale grid case:} The European grid scale in which all the synchronous machines are modeled and simplified as one single machine, provided with the characteristic expected from the overall system. Synchronous primary reserve deployment is in the order of $ \sim30 $ s \cite{ENTSOE.2016, hultholm2015optimal}. The frequency response is assumed to be the same that the European response analyzed by ENTSOE \cite{ENTSOE.2016}. Similarly as in the simplified model of the IEEE benchmark, the influence of synthetic inertia and IBFPR is evaluated.
\end{itemize}

\begin{table}[h]
	\caption{\label{tb:summary}: Summary of the simulated cases}
	\centering
	%% \tablesize{} %% You can specify the fontsize here, e.g., \tablesize{\footnotesize}. If commented out \small will be used.
	\begin{tabular}{lcc}
		\toprule
		\textbf{Cases}	& \multicolumn{2}{c}{\textbf{Assessment}} \\
		
		\midrule
		{}&IBFPR&	Synthetic Inertia\\
		\midrule
		Small scale grid &{} &{}\\		
		{	a) Simplified IEEE model}	& X &	X\\
		{	b) Extended IEEE model}&	X &{}\\	
		Large scale grid&	X&	X\\
		\bottomrule
	\end{tabular}
\end{table}
Therefore with the selected cases, the critical and nadir time are estimated through the simulation of different scenarios combining a range of imbalances and shares of non-synchronous generation. In order to assess Equation \eqref{eq:IBFPR}, a fit of the critical time as function of RoCoF is carried out. With the corresponding fit function for each case, Equation \eqref{eq:IBFPR} can be easily applied assuming that the inertia of the system is known and power imbalance can be calculated as: $ \frac{df}{dt}\frac{T_aP_{LOAD}}{f_0} $.

%\begin{equation}
%\Delta P(t)=\dfrac{df}{dt}\dfrac{T_aP_{LOAD}}{f_0}
%\end{equation}

%\begin{figure}[h]
%	\centering
%	\includegraphics[width=0.6\textwidth]{/method/method}
%	\caption{IBFPR methodology: \textbf{1}) Scenarios with different system's inertia and power imbalance are simulated \textbf{2}) Critical time and nadir time is obtained for each scenario \textbf{3}) A fit function relating critical time and RoCoF is obtained \textbf{4}) According to the RoCoF measured in the system, a power response is calculated and injected to the system.}
%	\label{fig:method}
%\end{figure}


\subsection{Simplified IEEE 9 bus Model}
\label{ssec:simpleieee}

As a first step to evaluate the impact of inverter based generation and power imbalances in the grid, the whole system is simplified as one single generating unit; neglecting all losses in the system (Transformers, transmission lines and generators) with the assumption that the mechanical output of the prime mover is the same than the electrical power output at generator terminals. Table \ref{tb:gridelements} provides a summary of the elements comprising the base model.\\




\begin{table}[h]
	\caption{\label{tb:gridelements}:Elements of the IEEE 9 bus model.}
	\centering
	%% \tablesize{} %% You can specify the fontsize here, e.g., \tablesize{\footnotesize}. If commented out \small will be used.
	\begin{tabular}{ccc}
		\toprule
		\textbf{}	& \textbf{Quantity}\\
		\midrule
		Buses		& 9			\\
		Transformers		& 3			\\
 		Transmission Lines			& 6 \\
		Generators			& 3 \\
 		Load			&  315 MW  \\
		\bottomrule
	\end{tabular}
\end{table}





Figure \ref{fig:ieeesimple} is the block representation of the swing equation \eqref{eq:swing}, it only differs in the fact that blocks representing the inverter based generation have been included. The mechanical power is represented by the output of a steam turbine governor model, which is used to represent the synchronous machine as depicted in Figure \ref{fig:gov}. When equilibrium is lost, the accelerating power is multiplied by the transfer function $ 1/(2HS) $, where $ H $ is the machine’s inertia constant and $ S $ is the machine’s power rating. From Equation \eqref{eq:swing} this product equals the derivative of frequency, therefore an integrator block is added to obtain the frequency response \cite{kundur1994power, john1994power, ogata1999ingenieria}. A feedback loop is added and an error signal obtained from the reference frequency so the synchronous machine can react as frequency deviates from nominal.

\begin{figure}[h]
	\centering
	\includegraphics[width=0.75\textwidth]{/method/ieee2}
	\caption{Simplified representation of the IEEE 9 bus model. Blocks linked by the solid line represent the conventional swing equation given by Equation \eqref{eq:swing}. Represented with dashed lines the respective frequency signals to the blocks of IBFPR and synthetic inertia, which add power to the system.}
	\label{fig:ieeesimple}
\end{figure}


The values of kinetic energy and time constants of a synchronous machines of 835 MVA were selected to represent the synchronous response, with the load of 315 MW the system acceleration time constant is 14 s, which is approximately today’s Europe acceleration constant \cite{ENTSOE.2016}. This is the base scenario where an 100\% synchronous generation is assumed . For the sake of evaluating the impact of the penetration of inverter based generation; the values of lower capacity generators were selected, diminishing the total system inertia.\\
\begin{figure}[h]
	\centering
	\includegraphics[width=0.75\textwidth]{/method/gov}
	\caption{Model of the general purpose governor for the representation of synchronous machines; where R is the turbine droop, P\textsubscript{ref} is the reference load at nominal frequency, T\textsubscript{1} is the governor delay, T\textsubscript{2} is the reset time constant, T\textsubscript{3} is the servo time constant, T\textsubscript{4} is the steam valve time constant and T\textsubscript{5} is the steam re-heat time constant \cite{Anderson.2002}.}
	\label{fig:gov}
\end{figure}

\begin{table}[h]
	\caption{\label{tb:timeconstant}: Typical generator values and governor settings as function of capacity \cite{Anderson.2002}}
	%\centering
	\begin{adjustbox}{width=\textwidth, center}
		
	%% \tablesize{} %% You can specify the fontsize here, e.g., \tablesize{\footnotesize}. If commented out \small will be used.
	\begin{tabular}{cccccccccccc}
		\toprule
		\textbf{Parameters}	& \multicolumn{11}{c}{\textbf{Generator Capacity (MVA)}} \\
		
		\midrule
		{} & 911&	835&	590&	410&	384&	192&	100&	75&	51.2&	35.29&	25 \\
		\midrule

		\textbf{T\textsubscript{1} (s)}&	0.1&	0.18&	0.08&	0.18&	0.22&	0.083&	0.09&	0.09&	0.2&	0.2&	0.2\\
		\textbf{T\textsubscript{2} (s)}&	0&	0.03&	0&	0&	0&	0&	0&	0&	0&	0&	0\\
		\textbf{T\textsubscript{3} (s)}&	0.2&	0.2&	0.15&	0.04&	0.2&	0.2&	0.2&	0.2&	0.3&	0.3&	0.3\\
		\textbf{T\textsubscript{4} (s)}&	0.1&	0&	0.05&	0.25&	0.25&	0.05&	0.3&	0.3&	0.09&	0.2&	0.09\\
		\textbf{T\textsubscript{5} (s)}&	8.72&	8&	10&	8&	8&	8&	0&	0&	0&	0&	0\\
		\textbf{Kinetic Energy (MWs)}&	2265&	2206.4&	1368&	1518.7&	1006.5&	634	&498.5&	464&	260&	154.9&	125.4\\
		\textbf{H (s)}&	2.486&	2.642&	2.319&	3.704&	2.621&	3.302&	4.985&	6.187&	5.078&	4.389&	5.016\\
		\textbf{P\textsubscript{max} (MW)}&	820&	766.29&	553&	367&	360&	175&	105&	75&	53&	36.1&	22.5\\
		\textbf{T\textsubscript{a}  (s)}&	14.381&	14.009&	8.686&	9.643&	6.390&	4.025&	3.165&	2.946&	1.651&	0.983&	0.796\\
		\bottomrule
	\end{tabular}
	\end{adjustbox}
\end{table}






Even though load imbalances up to 40\% were simulated in each inertia scenario, for estimation of the critical time the power capacity limit of the generators was disregarded. The negative imbalance was simulated by increasing the system load. 

\subsection{Extended IEEE 9 bus Model}

Since it is desired to compare the results obtained in Section \ref{ssec:simpleieee} against some model that takes into account the whole system components, losses and dynamics; An  extended representation of the IEEE 9 bus model was implemented in Simulink \cite{delavari2018simscape}. In this representation, simulations for different values of system inertia and load imbalance were performed, similarly as it was done with the simplified representation of the model. Figure \ref{fig:ieeeext} shows the extended IEEE 9 bus grid architecture with IBG added.\\
\begin{figure}[h]
	\centering
	\includegraphics[width=0.5\textwidth]{/method/IEEE_FPRmodel}
	\caption{One line diagram of the IEEE 9 bus model. The inverter based frequency response has been added at the same bus of the generating units.}
	\label{fig:ieeeext}
\end{figure}
In order to evaluate the validity of the equation describing the IBFPR needed to avoid ULFS, the IEEE model was modified with the insertion of ideal controlled power sources blocks, which were set up to inject power into the grid accordingly to the simulated scenario. Therefore, no means of frequency measurement were included and only IBFPR was assessed.
As it was done in Section \ref{ssec:simpleieee}, the total acceleration time constant of the system equals 14 s when there is no share of non-synchronous generation. Hence the same kinetic energy should be distributed among the three generators' rotating masses in the extended model as in the simplified representation. From Equation \eqref{eq:t_sys} it can be easily calculated that the system's kinetic energy with 14 s of acceleration time constant is 2205 MWs.
\begin{equation}
	\label{eq:t_sys}
	T_{sys}=2*E_{k} /P_{load}
\end{equation}

Due to the fact that inverter based generation reduces the system kinetic energy; for different levels of inverter based generation, the nominal capacity of the generators' values were kept constant and the inertia constant of each machine multiplied by the synchronous share factor $ fss $. The total kinetic energy of the system is the summation of all units.\\
%Equation 3 7
%E_k=2*(H*f_ss)*S
In order to start the simulations in steady state, a load flow calculation of the grid was carried out with the objective of calculating the initial conditions for the exciter and prime mover models.
Table \ref{tb:initial} summarizes the main values for setting system's initial conditions; acquired from the power flow tool provided by SIMSCAPE.


\begin{table}[h]
	\caption{\label{tb:initial}: Steady state initial conditions of the system}
	\centering
	%% \tablesize{} %% You can specify the fontsize here, e.g., \tablesize{\footnotesize}. If commented out \small will be used.
	\begin{tabular}{ccccc}
		\toprule
		\textbf{Bus number}	& \textbf{Bus Type}	& \textbf{Voltage (pu)}& \textbf{Active Power (MW)}& \textbf{Reactive Power (MVAr)}\\
		\midrule
		1		& Slack			& 1.04 $\phase{0^{\circ}} $     &    72.2    & 25.64    \\
		2		& PV			& 1.025 $\phase{9.83^{\circ}} $      & 163      & 8     \\
		3		& PV			& 1.025 $\phase{4.63^{\circ}} $     & 85       &    -9.41 \\
		5		& PQ			& 0.9949 $\phase{-4.42^{\circ}} $       &125       &  50    \\
		6		& PQ			& 1.01211 $\phase{-4.16^{\circ}} $      &   90     &  30   \\
		8		& PQ			& 1.0172 $ \phase{0.17^{\circ}} $       &  100     &   35   \\

		\bottomrule
	\end{tabular}
\end{table}


\subsubsection{IBFPR Representation}


The IBFPR was modeled as controlled current sources. These controlled sources inject active power according to the load imbalance and system inertia simulated. The continuous measurement of voltage is required in order to determine the amount of current needed to supply the requested power. The IBFPR will have symmetrical and balanced characteristics. Due to this reason, the magnitude and angle of the current phasor will be obtained from the positive sequence of the measured voltage. From the definition of complex power and voltage symmetrical components in three phase systems \eqref{eq:complex_p}, the positive sequence component of phase voltage and line current are obtained \cite{john1994power}.

\begin{equation}
\label{eq:complex_p}
S_{3\varphi}^1=3*V_{LN}^1*\bar{I}_{L}^1
\end{equation}

This equation is valid for RMS values of voltage and current in which $ S_{3\varphi}^1 $ is the positive sequence of the three phase complex power, $ V_{LN}^1 $ is the positive sequence of voltage line to neutral and $ \bar I_{L}^1 $ is the conjugated of the positive sequence line current; nevertheless the measured voltage values in Simulink are peak, then the equation for power and current become:

\begin{equation}
\label{eq:power_seq}
S_{3\varphi}^1=\dfrac{3*V_{LNpeak}^1*{\bar I_{Lpeak}}^1}{2}
\end{equation}


\begin{equation}
\label{eq:current_seq}
I_{Lpeak}^1=\dfrac{2*\bar{S}_{3\varphi}^1}{3*V_{LNpeak}^1}
\end{equation}
%Equation 3 10
%〖I_Lpeak〗^1=¯(((2*〖S_3ⱷ〗^1)/(3*〖V_LNpeak〗^1 )))
With the help of the \textbf{a} operator ($-0.5+j\sqrt{3}$ or $ 1\phase{120^{\circ}}) $ the values of the positive sequence component of phase voltage can be obtained. \\
From $ V_a +V_b+V_c=0$ and $ V_a^1=\frac{V_a+ aV_b+a^2 V_c}{3} $:
\begin{align*}
	 V_a^1 & =\dfrac{V_a+ aV_b-a^2 V_b-a^2}{3} \\
	& =\dfrac{V_a*(1-a^2)+ aV_b*(1-a)}{3}
\end{align*}

Since $V_{an}^1=\frac{V_a^1}{\sqrt{3}\phase{30^{\circ}}}$, $\sqrt{3}\phase{30^{\circ}}=1-a^2 $ and $ \sqrt{3}\phase{-30^{\circ}}=1-a $ then after some algebraic manipulation the expression for $ V_{an}^1 $ becomes:

\begin{equation}
	\label{eq:volt_seq}
	V_{an}^1=\dfrac{V_a-a^2 V_b}{3}
\end{equation}

With the obtained expressions  for the positive sequence of phase voltage \eqref{eq:volt_seq} and complex power \eqref{eq:power_seq}, the needed current \eqref{eq:current_seq} to supply the IBFPR related to the measured voltages can be implemented in Simulink as depicted in Figure \ref{fig:ieeeext_ibfpr}. The ramping function will last until the critical time is reached, afterwards, the IBFPR output will remain constant.

\begin{figure}[h]
	\centering
	\includegraphics[width=0.75\textwidth]{/method/extended_ibfpr}
	\caption{Implementation of IBFPR in the extended IEEE model. The dashed lines denote the interaction between signals from the power circuit to the control circuit and vice-versa. From the voltages readings of lines a-b and b-c the voltage $ V_{an} $ is calculated using Equation \eqref{eq:volt_seq}, then Equation \eqref{eq:current_seq} is implemented to calculate the current to be fed into the system using the complex power response obtained using Equation \eqref{eq:IBFPR} and the previous calculated value of $ V_{an} $.} 
	\label{fig:ieeeext_ibfpr}
\end{figure}

It must be noticed that when the IBFPR depicted in Figure \ref{fig:ieeeext_ibfpr} was implemented in Simulink, additional blocks were added in order to run the simulation, such blocks are a break algebraic loop just before the conjugate block. Additionally, a block to avoid division by zero was added at the output of the gain of $ 1 /3 $.

\subsection{Large Scale Case: Europe Power System}

Under normal operation ENTSOE has reported values of RoCoF in the range of 5-10 mHz/s for power outages of 1 GW in the current interconnected power system. If an imbalance event of more than 3 GW occurs with depleted primary reserve, extraordinary values of frequency and RoCoF might be reached. After serious disturbances the Continental European Power System has experienced RoCoF between 100 mHz/s and 1 Hz/s. Imbalances of 20\% or more along with RoCoF greater than 1 Hz/s have been determined by experience to be critical \cite{ENTSOE.2016}. ENTSOE has determined that the reference scenario (The loss of 3 GW generation with 150 GW load and 2\%/Hz self-regulation) in the interconnected operation, the influence of inverter based generation, and therefore the reduction of system inertia would not jeopardize system stability. Due to the expected increase of non-synchronous generation in the future, international power trade and renewables variability; ENTSOE estates in its future split reference scenario that the power system must be capable of withstanding imbalances greater than 40\% with RoCoF of 2 Hz/s or higher. Under these circumstances the resulting islands must avoid load shedding. Hence, the conditions of the split scenario are considered for further analysis.
\begin{figure}[h]
	\centering
	\includegraphics[width=0.6\textwidth]{/method/euro}
	\caption{Large scale grid derived from the simplified IEEE model. The synchronous response was modified to match the European reserve response and the effect of self-regulation was added.}
	\label{fig:euro}
\end{figure}

In order to fit the behavior of the system to the one modeled by ENTSOE, the synchronous representation in the simplified IEEE model shown in Figure \ref{fig:gov} was used as a base to emulate such behavior; this was done with the insertion of an additional block at the output of the governor model. With this approach, the primary power reserve can be easily tuned with the assistance of the Control System Tuner App available in MATLAB. The period of time of utmost interest for analysis is from the inception of the power imbalance and the nadir time. Therefore, the system must perform as similar as possible in this region compared to the ENTSOE reference, whereas after the nadir time, the disparity between responses can be neglected. In the European scale the reserves must be completely deployed within 30 s after the occurrence of the disturbance. 

\begin{figure}[h]
	\centering
	\includegraphics[width=0.5\textwidth]{/method/goveuro}
	\caption{Governor representation for the large grid scale case. The synchronous machine block represents the governor model used in Section \ref{ssec:simpleieee}. The Control System Tuner App sets the constants A, B, C and D of the additional block in the model in order to have a step response with a rise time of $ \sim30s $ by establishing an overshoot of 2\%  and a time constant of 8 seconds \cite{ogata1999ingenieria}.}
	\label{fig:goveuro}
\end{figure}




\subsubsection{System Parameters}

When a power system of $ n $ number of synchronous machines is assumed; having each of them a capacity $ S $ in MVA, a nominal power $ P_{nom} $ in MW and
supposing that each machine operates at a de-load factor $ dl $ of $ P_{nom} $; with an acceleration constant equal to $ T_{nom} $ then the number of machines $ n $, for the load $ P_{syncload} $ served by synchronous machines is:
%Equation 3 12
%n=P_(load_sync)/(P_nom*dl)
\begin{equation}
	n=\dfrac{P_{syncload}}{P_{nom}*dl}
\end{equation}
Then the time acceleration constant of the system $ T_{sys} $ can be obtained as follows:
\begin{align}
	T_{sys} &=\dfrac{\sum_{i=1}^nP_i*T_i}{P_{LOAD}}\nonumber  \\
	 &=\dfrac{nP_{nom}*T_i}{P_{LOAD}}\nonumber \\
	&=\dfrac{P_{syncload}*T_{nom}}{P_{LOAD}*dl}\nonumber\\
		&=\dfrac{Sync share*T_{nom}}{dl} \label{eq:tsyseuro}
\end{align}




In this sense the system's acceleration time constant can be calculated with a synchronous share of 100\%, resulting in $ T_{sys}=12.5s $   with values of $ T_{nom}=10s $  \cite{ENTSOE.2016, Anderson.2002}, and a de-load factor $ dl=0.8 $. Considering only the swing equation, it can be demonstrated that the RoCoF and therefore the frequency response of the system is only dependent on the percentage of load imbalance and the system acceleration time constant.
From the definition of RoCoF as $ \frac{df}{dt}=\frac{\Delta P*f_0}{2*E_k} $ and  $ T_{sys}=\frac{2*E_k}{P_{LOAD}} $ :

\begin{align}
	\dfrac{df}{dt} &=\dfrac{\Delta P*f_0}{P_{LOAD}*T_{sys}} \nonumber\\
	&=\dfrac{\Delta P_{pu}*f_0}{P_{LOAD}*T_{sys}}
	\label{eq:dfdpeuro}
\end{align}
In Equation \eqref{eq:dfdpeuro} the value of $ \Delta P_{pu} $ is the normalized value of power imbalance having as base power the value of load $ P_{LOAD} $. 


%%%%%%%%%%%%%%%%%%%%%%%%%%%%%%%%%%%%%%%%%%
\section{Results}
\label{sec:results}

\subsection{Analysis of Critical Time}


When the simplified and the extended models of the IEEE benchmark are compared for the estimation of the critical time, it is noticed a higher deviation in the low range of RoCoF. This due to the fact that in this range of RoCoF the critical time is long enough to allow the governor response activation of the respective synchronous machines representation. Therefore it can be stated that the simplifications made in the model have a greater influence on the results for low values of penetration of IBG and low power imbalances; in this sense, the simplifications become less significant as the RoCoF increases in such a manner that the activated synchronous primary reserve is not relevant in frequency support.In the range of higher values of RoCoF than 2 Hz/s, the critical time trend for the European grid-scale and the simplified IEEE model get closer each other as RoCoF increases.\\

%insert figure here

Therefore under high RoCoF conditions in any of the models, the primary reserve does not significantly counteract the frequency drop [16]. Figure 5 1 demonstrates that primary reserve can be neglected for determining the critical time when the combination of IBG and load imbalances would lead to high values of RoCoF; as RoCoF increases, the approximation of critical time as 1Hz/RoCoF narrows the difference with the results obtained from simulations [14].Nevertheless, such simplification applies to the simplified IEEE model and the European island. Hence, the influence of all the dynamics and machine components, such as
generator exciter and damping windings, seems to improve the critical time. Damping torque in equation [7, 8] was not considered for the IEEE simplified model; the inclusion of such may lead to more precise times when compared with the extended model.\\
\begin{table}[h]
	\caption{\label{tb:crtime}: Critical time for European-scale case given in seconds.}
	\centering
	%% \tablesize{} %% You can specify the fontsize here, e.g., \tablesize{\footnotesize}. If commented out \small will be used.
	\begin{tabular}{*9c}
		\toprule
		\textbf{IBG share (\%)}	& \multicolumn{8}{c}{\textbf{Load Imbalance (\%)}} \\
		\midrule
		{} & 3&	4&	5&	6&	7&	8&	9	&10 \\
		\midrule
		20&	-&	-&	6.081&	4.517&	3.629&	3.050&	2.638&	2.316\\
		40&	-&	6.226&	4.169&	3.215&	2.628&	2.222&	1.934&	1.705\\
		60	&7.142&	3.639&	2.623&	2.062&	1.698&	1.451&	1.263&	1.122\\
		80&	2.753&	1.744&	1.277&	1.018&	0.843&	0.722&	0.628&	0.559\\
		92&	1.109&	0.700&	0.514&	0.406&	0.338&	0.288&	0.252&	0.224\\
		95&	0.697&	0.436&	0.322&	0.254&	0.211&	0.179&	0.157&	0.140\\
		\bottomrule
	\end{tabular}
\end{table}






%Also it is then stated the need of a fast power response to avoid frequency collapse of islanded micro-grid or an electric island in the European scale.% Even assuming that power reserve can
%immediately fully activated after RoCoF reading, the 100 ms limitation is a constraint for high unbalanced islands with high penetration of IBG in the European case as demonstrated in the result section.

% As conclusion

%Additionally, the direct measurement of RoCoF in the 100 ms interval can lead to misleading readings [14]. In general, when penetration
%of IBG is higher than 90\%; for the 40\% imbalance an activation time between 30 and 50 ms would be needed to keep frequency within the allowed limits.
Due to the fact that the characteristics of the  European interconnected scenario provided by ENTSOE were assumed to be the same than the resulting islands after a severe event; the results for the large scale model can be understood as the behavior of the whole European system with bigger perturbations. The dimensioning scenario assumes a power imbalance of 3 GW, which corresponds to a 2\% of the 150 GW load [1]. If in future a bigger dimensioning case is utilized, then synchronous response would not be enough to balance the system before load shedding occurs. Table 5 1 exhibits the required time when the dimensioning scenario is increased up to 10\% for different IBG penetration.



\subsection{Analysis of Synthetic Inertia and Fast Power Reserve}

\subsubsection{Effect Synthetic Inertia on Frequency}

In this section are presented the results of the simulations carried out in the simplified IEEE model and the European model with the implementation of synthetic inertia control to the wind share of the IBG. The frequency nadir for such systems without any additional power support apart from synchronous response are illustrated in figures XXX and XXX. \\



Insert the figures here of no SI support\\

this as caption For the acceleration constant corresponding to the 80\% of the IBG, the frequency nadir reaches values lower than 49 Hz with power imbalances starting at $ \sim 7\% $.
this as caption At 80\% of IBG, frequecy nadir reaches 48.73 Hz with only $\sim 3\% $ of imbalance. \\


Insert figures with synthetic inertia here.
\\
In any of the cases UFLS is not avoided for all combination of variables (imbalance and acceleration constant). It can also be observed that frequency nadir under 49 Hz are reached for imbalances bigger than 14\% combined with shares of IBG above 80\%. simplified.

In the same manner than fig, in fig XXX the synthetic inertia is not enough for withstanding severe imbalances under high penetration of inverter based generators. At 80\% of IBG, the frequecy nadir reaches 48.89 Hz with an imbalance of 3\%. 

An enhanced performance is achieved in the simplified IEEE model as depicted in FIGIRE XXX. The reason behind this, is the faster response of the synchronous share present in the system, which jointly performs with the synthetic inertia to improve over all frequency response performance. In Figure XXX can be observed how power is deployed in a few seconds in the simplified IEEE model whereas in the European model the response is always limited to fully deployment in the order of $ \sim 30 $s.

Insert figure of the power response and freq response over time

Figure 4 9 and Figure 4 10 indicate the frequency response obtained of the system with an non-synchronous generation of 80\% for different load imbalances.  
In Figure 4 9 can be observed how the frequency drops below 49 Hz with a 10\% of imbalance, when no IBFPR or synthetic inertia is used as frequency support estrategy. In the same figure, the frequency responses for different levels of synthetic inertia is presented. 

It is noticed the improvement in FIG XXX in the response with the implementation of synthetic inertia with each contribution of wind energy to the non-synchronous generation. UFLS is avoided for every share of synthetic inertia, assuming that primary reserve takes place after SI. As the imbalance increases, the effectiveness of the synthetic inertia decreases, depending on the contribution of wind power to the inverter based generation. Figure 4 10 shows how a contribution of wind power of 40\% from the inverter based generation is capable of avoiding UFLS. Nevertheless the share of 80\% begins with a low rate of frequency decrement, the frequency sudenly drops below 49 Hz. This is due to the high amount of synthetic inertial power when compared to the load imbalance. This situation leads to UFLS after the 10 s because frequency has been sustained during that time by the synthetic inertial power. Since 10 seconds is the assumed time limit for exceeding nominal turbine power rate; the synthetic inertial power, which has a big contribution to counteract the power imbalance, is switched off as depicted in Figure 4 11. On the other hand, when a higher imbalance occurs and the synthetic inertial response is saturated, due to the limitation of 10\% of rated power, the mechanical power increases at 10 seconds, having a less severe impact the switching off of the inertial response.



\subsubsection{Effect of Power Ramp Response on Frequency}

The contribution from the ramping power in diminishing system RoCoF during the inception of the perturbation until the critical time was disregarded when \ref{eq:p_at_tcr} was calculated. Assuming an instant switching of the IBFPR at critical time, the frequency nadir would be 49 Hz. Nevertheless, a ramp power response was assumed instead. Therefore the calculated power ramp, when applied to the unbalanced system, commonly exhibits a frequency nadir higher than 49 Hz, due to the contribution of the ramping period. In this sense, it can be inferred that the longer the ramping period, the higher frequency nadir will be obtained. Here again the relevance of the prompt activation in time of the IBFPR. On the other hand, with the faster IBFPR activation, the ramp slope and the steady power output (Inverter based power reserve) can be diminished compromising frequency nadir.\\ %In order to assure stability a stepper ramp slope is required in order to meet the required power before critical time. That is achieved by changing equation 3-2 by the adjusted expression:
%Equation 5 1
%P_IBFPR (t)=∆P*((1-t_cr⁄t_nadir ))/((t_cr-t_d ) )*t
%Where td is the time delay needed to start the activation of IBFPR.
When a comparison is established beween all the calculated power ramp slopes in per unit  (pu), it is noted that with high penetration of non-synchronous power in the power system, the required power to ensure no UFLS have a consistent trend between the three models and a close proximity in the values for RoCoF in the range of 2 to 5 Hz/s is observed between both IEEE models. Such trends can be seen in Figure 5 2. A bigger amount of power ramp slope is needed in all the range of RoCoF for the European case. After inspecting \ref{eq:IBFPR}, it is noticed that the IBFPR is affected by the factor $ 1-t_{cr}⁄t_{nadir} $, then as nadir time increases, IBFPR increases as well. The nadir time for the European case, due to the action of the self-regulation and primary reserve deployment of 30 seconds, is in the range of 3-12 seconds (6 seconds for 80\% IBG penetration) whereas the nadir time for the simplified IEEE model is between 1-3 seconds.\\

Insert power ramp comparisson here

\subsubsection{Fast Power Reserve}


The required power ramp  to avoid load shedding has been found for both IEEE 9 bus models with a fast governor response and for the European-scale model  with conventional primary reserve response. Hence, the IBFPR at critical time which remain constant after the critical time, would be accounted as the fast power reserve. %Similarly as primary reserve estimations are performed considering the loss of generation at certain level; for fast power reserve the results are presented for the scenarios in which imbalance could reach even 40\%.

\begin{table}[h]
	\caption{\label{tb:crpowr}: Fast power reserve in per unit for European case. Power reserve expressed in pu with power load as the base}
	\centering
	%% \tablesize{} %% You can specify the fontsize here, e.g., \tablesize{\footnotesize}. If commented out \small will be used.
	\begin{tabular}{*9c}
		\toprule
		\textbf{IBG share (\%)}	& \multicolumn{8}{c}{\textbf{Load Imbalance (\%)}} \\
		\midrule
		{} & 3&	4&	5&	6&	7&	8&	9	&10 \\
		\midrule
		20&	-&	-&	0.025&	0.038&	0.049&	0.060&	0.070&	0.081\\
		40&	-&	0.016&	0.030&	0.041&	0.052&	0.063&	0.073&	0.083\\
		60&	0.005&	0.024&	0.035&	0.045&	0.056&	0.066&	0.077&	0.087\\
		80&	0.016&	0.028&	0.039&	0.049&	0.062&	0.070&	0.080&	0.09\\
		92&	0.021&	0.033&	0.043&	0.054&	0.064&	0.074&	0.084&	0.094\\
		95&	0.024&	0.035&	0.045&	0.055&	0.065&	0.075&	0.085&	0.096\\
		\bottomrule
	\end{tabular}
\end{table}

When IBFPR is implemented in all three cases, frequency drop below 49Hz is avoided for all the most of the RoCoF simulated, provided that enough IBFPR is available for the given imbalance. Figures XXX to XXX show how frequency is kept above 49 in all the cases.\\

Insert figures of IBFPR nadir here\\
%For imbalances higher than 3\% the fast power reserve has to cover mostly all the imbalance. The observed offset from the one to one relation is due to the load reduction caused by the frequency drop (load self-regulation). The dashed line represents the one to one proportion. In the case of the IEEE grid; the simplified model exhibits a permanent offset from the dashed line of around 0.05 pu, this due to the action of a faster governor response. Therefore, it can be said that in such scenario, the conventional governor response of synchronous machines would cover 5\% of the imbalances starting at 8\%. Since the values for nadir time are not independent from the imbalance in the Extended model, due to the non-linearity of the model, the calculated power reserve tend to equalize the whole imbalance. As imbalance increases, the critical time decreases and the nadir time increases, this makes the reducing factor tcr/tnadir from Equation 3-2 to decrease and narrows the difference between the calculated reserve and power imbalance.
%Similarly as critical time was presented in Table 5 1 the required fast power reserve in the European context is shown in Table 5 2 for extraordinary events


%As IBG increases the closer to the imbalance the fast power reserve needs to be. As it was demonstrated in the result section; with a faster conventional reserve, reduction in fast power reserve can be observed.

\subsection{Synchronizing effect, lack of damping torque and implications}

The diminishing of synchronous machines leads to a very weak network where synchronizing and damping torque, which are inherent characteristics of synchronous machines, are not enough to stabilize the system[7]. Although the implementation of IBFPR contributes keeping synchronous machine on step, oscillations in the speed/frequency response of the rotor are observed. This oscillations are created by the lack of damping torque which is provided mainly by the synchronous machines through damping windings, rotor field exciter and power system stabilizer[ref]. \\

Insert figures here

%Discussion
For the simplified IEEE model and the European island, only transfer functions describing an equivalent system governor were modeled. Hence in such approaches, the effect and dynamics of synchronous generator’s exciters and inter-machine interaction were not taken into account. The before mention factors influence greatly small signal stability [7, 8]
Even though the scoop of this thesis is to analyze the power-time characteristics needed to avoid frequency collapse; oscillations associated to big perturbation were observed but they could not been addressed by the simple injection of power to the system. Also in the IEEE model, when a penetration of 95\% of inverter based generation and 2\% of load imbalance are considered, UFLS is not reached but the system becomes unstable as shown in Figure 5 6 and Figure 5 5. From penetrations levels above 85\% complete frequency stability is not ensure with the injection of fast power reserve, only UFLS on the first 10 seconds approximately. Then the system becomes unstable with increasing amplitude oscillations.
It is important to note that ENTSOE in its EUROPEAN interconnected scenario, determined that there is no UFLS when an unbalance of 2\% with high contribution of non-synchronous generation occurs. Nevertheless, no inter-machine interaction was considered and therefore a similar effect as the one in Figure 5 6 could be experienced.%Discussion


%%%%%%%%%%%%%%%%%%%%%%%%%%%%%%%%%%%%%%%%%%
\section{Discussion}
\label{sec:discussion}
Conventional synchronous machines were found not to be able to ensure transient frequency stability in conditions of power imbalance exceeding 2\% in the case of the European model. The governor operation is by far too slow to constitute the unique solution for frequency support during the transient period. Inverter based fast power reserve would be needed to be activated in an extreme short time. Some extended capability to withstand power imbalances is observed with the faster governor response in the IEEE models. \\

Although unlikely to occur, the consideration of the uncertainty of synchronous reserves availability and possible power transmission congestion in future scenarios[], could lead to higher imbalances as nowadays. In order to avoid load shedding in penetration scenarios above 80\% of non-synchronous generation for and load imbalances up to 40\%, inverter based fast power reserve must be deployed over a time in the order of 100-500 ms, independently of grid size and primary reserve response. Nevertheless, today’s full power activation time of renewable sources without storage is in the range of 200 to 600 ms \footnote{Times required for the all measurement, signal transmission and processing and coordination of power electronic device's controls [ref]}. These activation times are adequate for power imbalances leading to values of RoCoF equal or less than 4 Hz/s as studied by ENTSOE in future scenarios [ref]. Hence with today’s frequency measuring time and power deployment from renewable sources; load shedding and possible total black outs would not be avoided. In scenarios with plenty of renewables and high expected imbalances, storage would be a key factor in order to avoid de-loading and curtailment of renewables, the fast activation times (<50 ms) and promising price reduction make storage a good strategy to provide power balancing in both over and under frequency cases [ref]. A comprehensive method for the estimation of the required time for the activation of the inverter based reserve was developed in this paper, with such method a frequency support strategy for inverters can be implemented for an expected level of imbalance as long as system inertia is monitored or estimated.\\

Synthetic inertia from wind turbines have a better performance when operated along with fast synchronous response systems as shown in section \ref{sec:res_si} with the simplified IEEE model. When synthetic inertia is implemented with slower response primary response such the case of the large scale of the European grid, it was not noted an appreciable improvement in regards of the frequency nadir. Therefore, synthetic inertia is not able by it self to regulate or restore frequency deviation[ref]. This predictable outcome limits its usage just for slowing down the frequency drop after a load event.  The influence of the gain $ K_i $ is fundamental, since the choice of a specific value can avoid load shedding just for certain range of imbalances. For instance, in Section REF it was demonstrated that the chosen value for $ K_i $ is adequate for imbalances of 10\% but as the imbalance increases to 15\%, the initial dependency of system to sustain the imbalance from the synthetic inertia makes the frequency to rapidly drop after 10 seconds, when the synthetic inertia has been removed. Figure XXX demonstrates that reducing the value of $ K_i $ from 10 to XXX stabilizes the system.\\
%Synthetic inertia could tackle with under-frequency phenomenon. With penetrations of 80\% or higher of IBG, and contribution of at least 20\% of the IBG from Wind turbines with synthetic inertia controls, UFLS is avoided up to imbalances of 10\% with a fast synchronous response (1-3 s in the IEEE modeled cases). For primary reserve deployment of 30 s (European model), synthetic inertia was not enough for avoiding UFLS.\\ Gain ki

If the dimensioning scenario for primary reserve is increased in the European context, synchronous response is not fast enough for imbalances higher than 2\%. Full activation time in the range of 0.14 and 2.75 seconds would need required for the fast power reserve for penetrations of non-synchronous generation above 80\% and imbalances between 3 and 10\%. Additionally, fast power reserve would be almost equivalent to the imbalance, meaning that almost no contribution from synchronous machines is obtained. Although UFLS is avoided in the extended model in scenarios with penetrations of non-synchronous generation above 85\% with injection of inverter based fast power reserve; total system stability is not ensured after a few seconds ($ \sim5 s $), due to the presence of un-damped oscillations provoked by the poor damping torque present in the system as consequence of synchronous share reduction. Even though the approach considered throughout this work was the fast power reserve deployment to avoid under-frequency load shedding. If the same frequency deviation from nominal is considered as the critical for the over-frequency case (51 Hz); the same values would be obtained for critical time and power response. The difference lies in the power flow direction, in this case power surplus should be removed from the grid or converter to another form of energy, like electrochemical storage. When a linear system is employed as the cases of the simplified IEEE model and the large scale scenario, no difference was found between the critical time for under and over-frequency. When the non-linearity of the system is included in the extended model, the critical times between under and over-frequency do not match as illustrated in Figure \ref{fig:res_over_under}\\

\begin{figure}[h]
	\centering
	\includegraphics[width=0.75\textwidth]{/result/over_under}
	\caption{Under and over-frequency events in the extended IEEE model, a difference of 45 ms between each critical time is found.}
	\label{fig:res_over_under}
\end{figure}

In general, similar behavior is exhibit from the different models and approaches, even though they differ considerably in size and complexity. Hence, simplified block representation of the power system seems to be a fair way to sketch overall system trends and responses. The difference in critical time estimation between a full grid simulation and a simplified model was calculated to differ between 20-35\%, such difference could be crucial in fast power reserve studies and therefore should be considered when precise applications are implemented. A comprehensive method for estimation of the inverter based fast power reserve and critical time were developed and proved through the implementation in the two cases. 



%%%%%%%%%%%%%%%%%%%%%%%%%%%%%%%%%%%%%%%%%%
\vspace{6pt}

%%%%%%%%%%%%%%%%%%%%%%%%%%%%%%%%%%%%%%%%%%
%% optional
%\supplementary{The following are available online at \linksupplementary{s1}, Figure S1: title, Table S1: title, Video S1: title.}

% Only for the journal Methods and Protocols:
% If you wish to submit a video article, please do so with any other supplementary material.
% \supplementary{The following are available at \linksupplementary{s1}, Figure S1: title, Table S1: title, Video S1: title. A supporting video article is available at doi: link.}

%%%%%%%%%%%%%%%%%%%%%%%%%%%%%%%%%%%%%%%%%%
\authorcontributions{For research articles with several authors, a short paragraph specifying their individual contributions must be provided. The following statements should be used ``conceptualization, X.X. and Y.Y.; methodology, X.X.; software, X.X.; validation, X.X., Y.Y. and Z.Z.; formal analysis, X.X.; investigation, X.X.; resources, X.X.; data curation, X.X.; writing--original draft preparation, X.X.; writing--review and editing, X.X.; visualization, X.X.; supervision, X.X.; project administration, X.X.; funding acquisition, Y.Y.'', please turn to the  \href{http://img.mdpi.org/data/contributor-role-instruction.pdf}{CRediT taxonomy} for the term explanation. Authorship must be limited to those who have contributed substantially to the work reported.}

%%%%%%%%%%%%%%%%%%%%%%%%%%%%%%%%%%%%%%%%%%
\funding{Please add: ``This research received no external funding'' or ``This research was funded by NAME OF FUNDER grant number XXX.'' and  and ``The APC was funded by XXX''. Check carefully that the details given are accurate and use the standard spelling of funding agency names at \url{https://search.crossref.org/funding}, any errors may affect your future funding.}
This research received no external funding
%%%%%%%%%%%%%%%%%%%%%%%%%%%%%%%%%%%%%%%%%%
%\acknowledgments{In this section you can acknowledge any support given which is not covered by the author contribution or funding sections. This may include administrative and technical support, or donations in kind (e.g., materials used for experiments).}

%%%%%%%%%%%%%%%%%%%%%%%%%%%%%%%%%%%%%%%%%%
\conflictsofinterest{The authors declare no conflict of interest.}

%%%%%%%%%%%%%%%%%%%%%%%%%%%%%%%%%%%%%%%%%%
%% optional
\abbreviations{The following abbreviations are used in this manuscript:\\

\noindent
\begin{tabular}{@{}ll}
ENTSOE &European Network of Transmission System Operators for Electricity\\
FPR &Fast Power Reserve\\
HVDC& High Voltage Direct Current\\
IEEE& Institute of Electric and Electronic Engineers\\
IBFPR& Inverter based Fast Power Reserve\\
IBG& Inverter Based Generation\\
PV &Photovoltaic\\
RoCoF& Rate of Change of Frequency\\
SI &Synthetic Inertia\\
UFLS& Under Frequency Load Shedding\\
WSCC& Western System Coordinated Council

\end{tabular}}

%%%%%%%%%%%%%%%%%%%%%%%%%%%%%%%%%%%%%%%%%%
%% optional
\appendixtitles{no} %Leave argument "no" if all appendix headings stay EMPTY (then no dot is printed after "Appendix A"). If the appendix sections contain a heading then change the argument to "yes".

\reftitle{References}

% Please provide either the correct journal abbreviation (e.g. according to the “List of Title Word Abbreviations” http://www.issn.org/services/online-services/access-to-the-ltwa/) or the full name of the journal.
% Citations and References in Supplementary files are permitted provided that they also appear in the reference list here.

%=====================================
% References, variant A: external bibliography
%=====================================
%\printbibliography
\externalbibliography{yes}
%\usepackage{biblatex}
%\addbibresource{Thesisref_LATEX.bib}
\bibliography{Thesisref_LATEX}
% The following MDPI journals use author-date citation: Arts, Econometrics, Economies, Genealogy, Humanities, IJFS, JRFM, Laws, Religions, Risks, Social Sciences. For those journals, please follow the formatting guidelines on http://www.mdpi.com/authors/references
% To cite two works by the same author: \citeauthor{ref-journal-1a} (\citeyear{ref-journal-1a}, \citeyear{ref-journal-1b}). This produces: Whittaker (1967, 1975)
% To cite two works by the same author with specific pages: \citeauthor{ref-journal-3a} (\citeyear{ref-journal-3a}, p. 328; \citeyear{ref-journal-3b}, p.475). This produces: Wong (1999, p. 328; 2000, p. 475)



\end{document}
